\documentclass[twoside,colorback,accentcolor=tud4c,11pt]{tudreport}
\usepackage{ngerman}
\usepackage[utf8]{inputenc} 
\usepackage[T1]{fontenc}
\usepackage{siunitx}
\usepackage{hyperref}
\usepackage{units}
\usepackage{upgreek}
\usepackage{biblatex}
\usepackage{graphicx}
\usepackage{float}
\usepackage{subfigure}
\usepackage[figure]{hypcap}
\usepackage{isotope}

\title{Versuch 4.3: Kühlen und Fangen von Rubidium-Atomen in einer MOT}
\subtitle{	\begin{tabular}{p{8cm}ll}
Dominik Pfeiffer   &   Jonas Fischer \\ Matrikelnummer: 2913632  &   Matrikelnummer: mnr       \\ email: \textaccent{ dominik@diepfeiffers.de} & email: \textaccent{email}  
			\end{tabular} }
\subsubtitle{ \\Versuchsbetreuung : Dominik Schäffner \\ Datum der Durchführung: 26.06.2017 \\ Abgabetermin:17.07.2017    }
\institution{Praktikum für Fortgeschrittene}
\sponsor{Hiermit erklären wir, dass wir die vorliegende Arbeit bzw. Leistung eigenständig, ohne fremde Hilfe und nur unter Verwendung der angegebenen Hilfsmittel angefertigt haben. Alle übernommenen Textstellen aus der Literatur beziehungsweise dem Internet sind als solche kenntlich gemacht. Diese Arbeit hat in gleicher oder ähnlicher Form noch keiner Prüfungsbehörde vorgelegen. \\\\ 
\begin{tabular}{lp{2em}lp{2em}l}
 \hspace{4cm}   && \hspace{4cm}  && \hspace{4cm}
 \\\cline{1-1}\cline{3-3}\cline{5-5}
 Ort, Datum     && Dominik Pfeiffer && Jonas Fischer
\end{tabular}  }


\dedication{}
\lowertitleback{}
\listfiles
    
\begin{document}

\maketitle 

\tableofcontents


\chapter{Einleitung und Ziel des Versuchs}
Ziel dieses Versuches ist es schrittweise eine magnetooptische Falle (MOT) aufzubauen, $\isotope[85]{Rb}$-Atome in dieser zu fangen und die Temperatur in verschiedenen Phasen der MOT zu bestimmen. Hierfür wird zunächst das zum Kühlen verwendete Lasersystem stabilisiert, dann die Ladephase der MOT überprüft und schließlich die zur Temperaturbestimmung notwendigen Messungen durchgeführt.

\chapter{Physikalische Grundlagen}
Zunächst sollen der Aufbau und dessen wichtigste Komponenten, sowie die Theorie hinter der Laserkühlung bis in den unteren Mikrokelvinbereich und das räumliche Fangen in der Falle näher betrachtet werden.
\section{Aufbau und wichtige Komponenten}
\section{Theorie zur Laserstabilisierung}
\section{Theorie zur MOT} 	
\chapter{Versuchsdurchführung und Auswertung}
\section{Stabilisierung des Lasersystems}
\section{Ladephase der MOT}
\section{Temperaturbestimmung in der MOT}
\subsection{Temperatur der MOT}
\subsection{Temperatur in der optischen Melasse}
\chapter{Diskussion}
\chapter{Messdaten}


\chapter{Anhang}





		

\renewcommand{\bibname}{Literaturverzeichnis}
\begin{thebibliography}{Bak89}



\end{thebibliography} 	



\end{document} 